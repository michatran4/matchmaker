\documentclass[12pt]{article}

\title{matchmaker-tm}
\author{Michael Tran}
\date{\today}

\begin{document}
\maketitle

\section{Outline}
This outlines the features that matchmaker-tm will have.

\subsection{Gathering questions}
People will be matched based on their responses to high quality questions. These
questions will have at least two answers that can be distinctive of
compatibility.

\subsection{Gathering answers}
Answers should be sorted on a spectrum. Those that choose the most left answer
should be very incompatible with those that choose the most right answer. This
could be subjective, so I will have this spectrum crowdsourced.

\subsection{Interface}
Representatives will generate a token that will act as the user's login. This
token will be used to access the panel where answers will be selected. All
questions will already be final. The token will be null once the survey is
submitted.

subsection{Computing answers}
After a solid amount of data is collected, each user will have their answers
compared against other user's answers. Each answer has a difference from 0 to 1.
A larger difference represents incompatibility, and a smaller difference
represents compatibility. However, one question isn't indicative of everything,
so the averages of the differences are calculated and laid out onto a
spreadsheet. A person that has a smaller difference for a preference for another 
person is more compatible with them.

After averages are compiled, preferences will be sorted from the smallest
difference to the largest difference. A stable matching algorithm will then be
applied to matchmake users. An additional filter will include a preference for
gender.

\subsection{Data required}
The data I will need is each person's name, contact information, and answers to
questions. This can easily be stored in a database. Information may be encrypted
with GPG.

\section{Spectrum Analysis}
Answers will be laid out on a spectrum, and this will allow for some easy math.
This approach works because if a question has only two choices, then it implies
that the two choices are polar opposites of each other. This will be perfectly
represented with a large difference of 1 between adjacent choices. A question
with two choices will have the biggest difference, and if Person A chooses the
first choice and Person B chooses the second choice, the room for compatibility
between the two decreases. A question with one hundred choices will have less
of an impact, as differences between adjacent choices will only be 0.01.

\subsection{Q/A Formatting}
Questions and answers will be formatted like the following example:

\begin{verbatim}
    What animal do you like as a pet?
    Cats | Birds | Dogs
\end{verbatim}

Vertical bars will act as delimiters. Answers will be loaded from this format,
and proposed answers will be submitted in this format. Ultimately, I will gather
data from all the spectrums and create the most appropriate one.

\subsection{Creating an interface for surveying content}
Vanilla javascript and html should allow for dragging and dropping answers into
a row, and javascript automatically sorts the answers. This interface will
allow for ease of creating survey questions and submission of the answers.
Collecting answers will be done before the matchmaking is open to everyone.

\subsection{Revising survey content}
Questions should be revised after they are made. This is done by providing the
same question maker, except with a populated form; questions and answers will be
pre-filled. This allows for adding of answers and possible revision of the
question.

Populating the form is done with providing a file, and the form will be
generated with nodejs. I will get diffs manually to accept revisions to the
final survey.

\section{Answering survey questions}
Proposed spectrums will be compared and I will ultimately decide upon the best
one for people to answer to.

\subsection{Loading questions}
As stated, this is the format:
\begin{verbatim}
    What animal do you like as a pet?
    Cats | Birds | Dogs
\end{verbatim}

A simple html form will load up these sets of data and have users click their
answer. This form will be generated on my machine.

\subsection{Output of answers}
Questions will be in the same order for every person. Additionally, the survey
will have a final revision before it is released. Output is simply a list of
numbers. The list of numbers will be the index of an answer for each question.
This list of numbers will be in order, so question numbers will correspond.

\section{Data Computation}
I will be using Java for data computation.
First, the survey file will be read. The information needed is the number of
answers for each question.

Then, each users' choice is an index of an answer for each question. The choice
will have a calculated value that is the index divided by the difference between
the number of answers and 1. Answer indices start at 0, so this will allow for
answers on opposite ends of a spectrum to have a difference of 1.

Users that participate first in the matchmaker will be prioritized. This can be
done by making the file names alphabetical as Java will read the files
alphabetically. The users created from the files are added to the queue in
order.

\subsection{Data Output}
The final data spreadsheet will be a table with each person's compatibility
with one another. This spreadsheet will be kept private. CSVs are supported by
regular sheet readers like Excel and Calc.

It is probably one's best interest to just display matched results instead of
giving out information regarding who one was most compatible with. Each
individual will know whom he/she matched with after the stable matchmaking
algorithm is applied.

\end{document}
