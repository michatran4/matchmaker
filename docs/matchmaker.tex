\documentclass[12pt]{article}

\title{matchmaker-tm}
\author{Michael Tran}
\date{\today}

\begin{document}
\maketitle

\section{Outline}

This outlines the features that matchmaker-tm will have.

\subsection{Gathering questions}
People will be matched based on their responses to high quality questions. These questions should have at least two answers that can be distinctive of preferences.

\subsection{Gathering answers}
Answers should be sorted on a spectrum. Those that choose the most left answer should be very incompatible with those that choose the most right answer. This could be subjective, so I will have this spectrum crowdsourced. Users that participate in answering questions should provide their own question and answers.

\subsection{Interface}
Representatives will generate a token that will act as the user's login. This token will be used to access the panel where answers will be selected, and where a custom question will be submitted for review to add on to the current list of questions.

\subsection{Computing answers}
After a solid amount of data is collected, each user will have their answers compared against other user's answers. Each answer has a difference from 0 to 1. A larger difference represents incompatibility, and a smaller difference represents compatibility. However, one question isn't indicative of everything, so the averages of the differences are calculated and laid out onto a spreadsheet.

After averages are compiled, preferences will be sorted from the smallest difference to the largest difference. A stable matching algorithm will then be applied to matchmake users.

\section{Spectrum Analysis}
Answers will be laid out on a spectrum, and this will allow for some easy math. This approach works because if a question has only two choices, then it implies that the two choices are polar opposites of each other. This will be perfectly represented with a large difference of 1 between adjacent choices. However, a question with one hundred choices will have less of an impact, as differences between adjacent choices will only be 0.01.

\subsection{Q/A Formatting}
Questions and answers will be formatted like the following example:
x
\begin{verbatim}
    What animal do you like as a pet?
    Cats | Birds | Dogs
\end{verbatim}

Vertical bars will act as delimiters. Answers will be loaded from this format, and proposed answers will be submitted in this format. Ultimately, I will gather data from all the spectrums and create the most appropriate one.

\subsection{Creating an interface for surveying content}
Vanilla javascript and html should allow for dragging and dropping answers into a row. This interface will allow for ease of creating survey questions. Collecting answers will be done before the matchmaking is open to everyone.

Submitting answers just combines the questions and answers together with a newline. Javascript already sorts the answers, so they are all looped through in order.

\subsection{Revising survey content}
Questions should be revised after they are made. This is done by providing the same question maker, except with a populated form; questions and answers will be pre-filled. This allows for adding of answers and possible revision of the question.

Populating the form is done with providing a file. I will get diffs manually to accept revisions.

\section{Answering survey questions}
Proposed spectrums will be compared and I will ultimately decide upon the best one for people to answer to.

\subsection{Loading questions}
As stated, this is the format:
\begin{verbatim}
    What animal do you like as a pet?
    Cats | Birds | Dogs
\end{verbatim}

A simple html form will load up these sets of data and have users click their answer.
This form will be generated on my machine.

\subsection{Output of answers}
Questions will be in the same order for every person. Additionally, the survey will have a final revision before it is released. This could simply return a list of numbers. The list of numbers will be the answers in order, and each answer is an answer to the question in its corresponding index.

Data will be manually entered with manual diff comparisons.

\section{Data Computation}


\end{document}
