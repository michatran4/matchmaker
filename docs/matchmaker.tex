\documentclass[12pt]{article}

\title{matchmaker-tm}
\author{Michael Tran}
\date{\today}

\begin{document}
\maketitle

\section{Outline}
This outlines the features that matchmaker-tm will have.

\subsection{Gathering Questions and Answers}
People will be matched based on their responses to high quality questions. These
questions will have at least two answers that can be distinctive of
compatibility.

Answers should be sorted on a spectrum. Those that choose the most left answer
should be very incompatible with those that choose the most right answer. This
could be subjective, so I will have this spectrum crowdsourced.

The answers should be sortable on the spectrum. \textbf{The containers should
not be sortable}; this will create conflicts between where the dragged object
should be placed.

\subsection{Representative Interface}
Representatives will be allowed to edit the Google Form containing the survey.
Representatives will authenticate survey partcipiants by allowing the user's
student ID to have access to the survey. All questions will already be final at
the time of the surveying. Users will have only one survey submission.

\subsection{Computing answers}
After the matchmaker has been available to the public for a good amount of time
and a solid amount of data is collected, each user will have their answers
compared against other user's answers. Answers have a difference from 0 to
1. A larger difference represents incompatibility, and a smaller difference
represents compatibility. This is done by comparing where the answers are
located on the spectrum. As stated, \textit{those that choose the most left answer
should be very incompatible with those that choose the most right answer.}

However, one question isn't indicative of everything, so the averages of the
differences are calculated and laid out onto a spreadsheet. A person that has a
smaller difference for a preference for another person is more compatible with
them. Additionally, question weights will be included for questions that are
mainly for fun and don't truly indicate incompatibility.

After averages are compiled, a user's preferences will ultimately be sorted from
the smallest difference to the largest difference.

\subsection{Data required}
In the survey form, each person will need to provide:
\begin{itemize}
    \item one's name,
    \item one's contact information for the matchmaker results to be reported to,
    \item one's contact information to be sent out to who one matched with,
    \item and the selected answers to survey questions.
\end{itemize}

\section{Spectrum Creation / Analysis}
Answers will be laid out on a spectrum, and this will allow for some easy math.
Questions with less choices imply that adjacent choices are more conflicting and
indicative of incompatibility. However, questions with many more choices will
have adjacent answer choices that have less of a difference between one
another.

\subsection{Spectrum Examples}
A question with two choices will have the biggest difference, as if Person A chooses the
first choice and Person B chooses the second choice, there is lower compatibility.
On the other hand, a question with one hundred choices will have less of an
impact, as differences between adjacent choices will only be 0.01. If Person A
chooses the first choice and Person B chooses the second choice (right next to
the first choice), then there is still a chance both can be very compatible.

\subsection{Question Weights}
Question weights can be included. These are multipliers. Multiplying a
difference by a number greater than 1 exaggerates the difference. Multiplying a
difference by a number less than 1 minimizes the difference. A difference that
is exaggerated truly indicates incompatibility, while a difference that is
minimized means that the answer to the question is mostly just for fun.

The following format will accept question weights:

\begin{verbatim}
    (Decimal) [question]
\end{verbatim}

An opening parenthesis is the first character provided, with the decimal
following it, and a closing parenthesis with a space before the question. This
is to be as unambiguous as possible.

\subsection{Output Format}
The output of questions and answers will be formatted like the following
example:

\begin{verbatim}
    What animal do you like as a pet?
    Cats | Birds | Dogs
\end{verbatim}

Vertical bars will act as delimiters. Answers can be loaded from this format,
and proposed answers will be submitted in this format. Ultimately, I will gather
data from all the spectrums and create the most appropriate one.

\subsection{Creating the spectrum interface}
Vanilla javascript and html should allow for dragging and dropping answers into
a row, and javascript automatically sorts the answers. This interface will
allow for ease of creating survey questions and submission of the answers.
Collecting answers will be done before the matchmaking is open to everyone.

Questions should be revised by others after they are made. Questions and answers
can populate the form with a previous answer file. This allows for adding of
answers and possible revision of the question.

\section{Generation and Usage of the Survey}
I will get diffs manually. Proposed spectrums will be compared and I will
ultimately decide upon the best one for people to answer to.

The survey form's metadata will be generated with nodejs on my own client. The
additional vertically formatted survey is for ease of entering questions and
answers into a Google Form.

\subsection{Output of answers}
From the Google Sheet integrated with the Google Form, the index for each
answer will need to be found from a Map. A list of indices will be made, and it
should be in order so that questions correspond.

\section{Data Computation}
I will be using Java for data computation. First, the survey file will be read.
The information needed is the number of answers for each question, and each
question's weight. The weights can be included in the file containing the list
of numbers for each question.

Then, each users' choice is an index of an answer for each question. The choice
will have a calculated value that is the index divided by the difference between
the number of answers and 1. Answer indices start at 0, so this will allow for
answers on opposite ends of a spectrum to have a difference of 1. Differences
will then be multiplied by the weight of the question.

\subsection{Data Output}
The final data spreadsheet will be a table with each person's compatibility
with one another. This spreadsheet will be kept private. CSVs are supported by
regular sheet readers like Excel and Calc.

\end{document}
