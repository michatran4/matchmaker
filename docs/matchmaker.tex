\documentclass[12pt]{article}

\title{matchmaker}
\author{Michael Tran}
\date{\today}

\begin{document}
\maketitle

\section{Outline}
This outlines the features of matchmaker.

\subsection{Gathering Questions and Answers}
People will be matched based on their responses to high quality questions. These
questions will have at least two answers that can be distinctive of
compatibility.

Answers should be sorted on a spectrum. Those that choose the topmost answer
should be very incompatible with those that choose the bottommost answer. This
could be subjective, so all questions will be crowdsourced.

All questions will already be final at the time of the surveying. Users will
have only one survey submission.

\subsection{Computing answers}
After the matchmaker closes submissions, each user will have their answers
compared against other user's answers. Answers have a difference from 0 to
1. A larger difference represents incompatibility, and a smaller difference
represents compatibility. This is done by comparing where the answers are
located on the spectrum. As stated, \textit{those that choose the topmost answer
should be very incompatible with those that choose the bottommost answer.}

However, one question isn't indicative of everything, so the averages of the
differences are calculated and laid out onto a spreadsheet. A person that has a
smaller difference for a preference for another person is more compatible with
them. Additionally, question weights will be included for questions that are
mainly for fun and don't truly indicate incompatibility.

After averages are compiled, a user's preferences will ultimately be sorted from
the smallest difference to the largest difference.

\subsection{Data required}
In the survey form, each person will need to provide:
\begin{itemize}
    \item one's name,
    \item one's student ID,
    \item one's gender,
    \item one's email for the matchmaker results to be reported to,
    \item one's public contact information to be sent out to who one matched with,
    \item and the selected answers to survey questions.
\end{itemize}

\section{Survey Creation}
Answers are thought of to be on a 'spectrum.' Questions with less choices imply
that adjacent choices are more conflicting and indicative of incompatibility.
However, questions with many more choices will have adjacent answer choices that
have less of a difference between one another.

\textbf{Questions cannot start with a parenthesis. These are reserved for adding
weights to questions, later.}

\subsection{Spectrum Examples}
A question with two choices will have the biggest difference, as if Person A
chooses the first choice and Person B chooses the second choice, there is lower
compatibility. On the other hand, a question with one hundred choices will have
less of an impact, as differences between adjacent choices will only be 0.01. If
Person A chooses the first choice and Person B chooses the second choice (right
next to the first choice), then there is still a chance both can be very
compatible.

\subsection{Question Weights}
Question weights can be included to act as multipliers. Multiplying a difference
by a number greater than 1 exaggerates the difference, while multiplying a
difference by a number less than 1 minimizes the difference. A difference that
is exaggerated truly indicates incompatibility, while a difference that is
minimized means that the answer to the question is mostly just for fun.

This means that if a question is truly indicative of compatibility, then it
should have a weight that is greater than 1. However, if a question is mostly
for fun, then it should have a weight less than 1. Else, the default weight is 1
for a question and it doesn't need to be modified.

The following format will accept question weights:

\begin{verbatim}
    (Decimal) [question]
\end{verbatim}

An opening parenthesis is the first character provided, with the decimal
following it, and a closing parenthesis with a space before the question. This
is to be as unambiguous as possible.

\subsection{Output Format}
The output of questions and answers will be formatted like the following
example:

\begin{verbatim}
    What animal do you like as a pet?
    Cats
    Birds
    Dogs

    What is your favorite type of number?
    Even
    Odd
\end{verbatim}

The matchmaker organizer will determine the most appropriate questions and
answers and their ordering.

\section{Generation and Usage of the Survey}
The survey form's metadata will be generated with nodejs on the organizer's
client. This metadata will need to be moved into a directory for matchmaking.

\subsection{Output of answers}
From the Google Sheet integrated with the Google Form, the index for each
answer will need to be found from a Map. A list of indices will be made, and it
should be in order so that questions correspond.

Header information should also be specified. The CSV file may have headers in
different columns depending on who creates the survey. Have a file specifying
it so the data regarding identity can be parsed appropriately.

\section{Data Computation}
Java will be used for data computation. First, the survey file will be read.
The information needed is the number of answers for each question. Then, the
weights file will be read indicating each question's weight.

Each users' choice is an index of an answer for each question. The choice will
have a calculated value that is the index divided by the difference between the
number of answers and 1. Answer indices start at 0, so this will allow for
answers on opposite ends of a spectrum to have a difference of 1. Differences
will then be multiplied by the weight of the question.

\subsection{Data Output}
The final data spreadsheet will be a table with each person's compatibility with
one another. This spreadsheet will be kept private. CSVs are supported by
regular sheet readers like Excel and Calc.

In addition, individuals' results will be saved to separate files that can be
sent out.

\end{document}
